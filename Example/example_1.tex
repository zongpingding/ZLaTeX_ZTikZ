% show angle 
\tikzset{
    right angle quadrant/.code={
        \pgfmathsetmacro\quadranta{{1,1,-1,-1}[#1-1]}     % Arrays for selecting quadrant
        \pgfmathsetmacro\quadrantb{{1,-1,-1,1}[#1-1]}},
    right angle quadrant=1, % Make sure it is set, even if not called explicitly
    right angle length/.code={\def\rightanglelength{#1}},   % Length of symbol
    right angle length=2ex, % Make sure it is set...
    right angle symbol/.style n args={3}{
        insert path={
            let \p0 = ($(#1)!(#3)!(#2)$) in     % Intersection
                let \p1 = ($(\p0)!\quadranta*\rightanglelength!(#3)$), % Point on base line
                \p2 = ($(\p0)!\quadrantb*\rightanglelength!(#2)$) in % Point on perpendicular line
                let \p3 = ($(\p1)+(\p2)-(\p0)$) in  % Corner point of symbol
            (\p1) -- (\p3) -- (\p2)
        }
    }
}

% main code
\begin{tikzpicture}[scale=.75, >=Latex]
    \ShowGrid[step=1, color=gray, opacity=.5]{(-8, -8); (8, 8)}
    \ShowXYAxis{8}{8}
    % curve
    \ContourPlot[-8:8; -8:8]{x**2/3-y**2/4-1}
    \ContourPlot[-8:8; -8:8][dashed, red]{(x-3.05)**2 + (y-2.18)**2-4.76}
    % points and lines
    \coordinate (A) at (5.69, 6.26);
    \coordinate (B) at (-5, 0);
    \coordinate (C) at (5, 0);
    \coordinate (D) at (3.05, 2.18);
    \draw (A) -- (B) -- (C) -- cycle;
    % angle and 3-lines
    \draw [blue,right angle symbol={A}{1.94,4.06}{D}]; % F
    \draw [blue,right angle symbol={A}{5.21,1.94}{D}]; % E
    \draw [blue,right angle symbol={B}{3.05,0}{D}];    % G
    \draw (D) -- (1.94, 4.06);
    \draw (D) -- (5.21, 1.94);
    \draw (D) -- (3.05, 0);
    \ShowPoint[type=*, color=red]{(A); (B); (C); (D)}[$A$; $F_1$; $F_2$; $o$][below right]
\end{tikzpicture}