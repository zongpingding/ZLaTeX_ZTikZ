\chapter{zTikZ部分源码说明}
\section{Set up Env}
\subsection{Set up}
\begin{source}
\RequirePackage{l3sys-shell}
\NewDocumentCommand\ztikzMkdir{m}{ \sys_shell_mkdir:n {#1} }
\ExplSyntaxOff
\begingroup\newif\ifztikz
\IfFileExists{./ztikz_output/scripts/sympy_script.py}{}{\ztikztrue}
\ifztikz
% create folders and hash file
\ztikzMkdir{ztikz_output/}
\ztikzMkdir{ztikz_output/gnuplot_data/}
\ztikzMkdir{ztikz_output/python_data/}
\ztikzMkdir{ztikz_output/scripts/}
\ztikzMkdir{ztikz_output/tikz_data/}
\ztikzMkdir{ztikz_output/mma_data/}
\immediate\write18 {touch ztikz_output/ztikz.hash}

% writing source code
\begin{filecontents}[noheader]{./ztikz_output/scripts/python_script.py}
from numpy import *
from scipy import *


# ==> echo to file
Float_res = 2**10

# ==> write output
filename = "./ztikz_output/python_data/PyFloat.out"
with open(filename, 'w') as file: 
    file.write(str(Float_res)+"\n")
\end{filecontents}

\begin{filecontents}[noheader]{./ztikz_output/scripts/sympy_script.py}
from sympy import *


# ==> symbols declearation
x, y, z, u, v, t = symbols('x y z u v t') 

# ==> echo to file
F_res = integrate(sin(x)/x, (x, -oo, oo))

# ==> write output
filename = './ztikz_output/python_data/sympy.out'
with open(filename, 'w') as file: 
    file.write(latex(F_res)+"\n")
\end{filecontents}

\begin{filecontents}[noheader]{./ztikz_output/scripts/plot.gp}
set table "./ztikz_output/gnuplot_data/gnu_data.table"
set format "%.5f"
set samples 100


# ==> specific a 2d function
f(x) = x
set xr [-6:-1]
plot f(x)
\end{filecontents}

\begin{filecontents}[noheader]{./ztikz_output/scripts/contour_plot.gp}
set table "./ztikz_output/gnuplot_data/gnu_data.table"
set samples 100,100
set isosamples 100,100
set format "%.5f"
set cont base
set cntrparam levels discrete 0,0
unset surface


# ==> contour plot a function
set xr [-4:4]
set yr [*:*]
set zr [*:*]
f(x, y) = y-3*sin(1/x)
splot f(x,y)
\end{filecontents}

\begin{filecontents}[noheader]{./ztikz_output/scripts/param_plot.gp}
set table "./ztikz_output/gnuplot_data/gnu_data.table"
set format "%.5f"
set samples 100
set parametric


# ==> specific a parametric function
set trange [0:2*pi]
plot sin(t), cos(t)  
\end{filecontents}

\begin{filecontents}[noheader]{./ztikz_output/scripts/polar_plot.gp}
set table "./ztikz_output/gnuplot_data/gnu_data.table"
set format "%.5f"
set samples 100
set polar


# ==> polar plot, default varible name 't'
set trange [0:12*pi]
plot t
\end{filecontents}
\fi\endgroup
\ExplSyntaxOn
\end{source}

\subsection{config}
\begin{source}
% data or picture path
\tl_const:Nn \g__ztikz_gnu_path_tl      {./ztikz_output/gnuplot_data} 
\tl_const:Nn \g__ztikz_python_path_tl   {./ztikz_output/python_data} 
\tl_const:Nn \g__ztikz_scripts_path_tl  {./ztikz_output/scripts}
\tl_const:Nn \g__ztikz_tikz_path_tl     {./ztikz_output/tikz_data}
\tl_const:Nn \g__ztikz_mma_path_tl      {./ztikz_output/mma_data}

% index of output data by gnuplot
\int_new:N   \g__gnu_data_index_int
\int_new:N   \g__sympy_index_int
\int_new:N   \g__picture_index_int
\int_new:N   \g__mma_index_int
\int_new:N   \g__tikz_env_index_int
\int_gadd:Nn \g__mma_index_int {1}

% bool to control \PlotPrecise cmd
\bool_new:N \g__plot_precise_bool
\bool_new:N \g__contour_precise_bool
\bool_new:N \g__param_precise_bool
\bool_new:N \g__polar_precise_bool
\bool_new:N \g__hash_change_bool
\bool_gset_true:N \g__hash_change_bool

% io read/write interface
\ior_new:N \g__file_read_ior
\tl_new:N  \g__file_content_tl
\end{source}

\section{tikz}
\subsection{Cache mechanism}
\begin{source}
% generate md5 hash (history) of a file
\ior_new:N \g__ztikz_file_ior
\seq_new:N \g__ztikz_file_hash_seq
\cs_new_protected:Nn \ztikz_file_read_lines_cs:n {
    \ior_open:Nn \g__ztikz_file_ior {#1}
    \seq_gclear:N \g__ztikz_file_hash_seq
    \ior_str_map_inline:Nn \g__ztikz_file_ior
      {
        \seq_gput_right:Nx \g__ztikz_file_hash_seq
          { \tl_trim_spaces:n {##1} }
      }
    \ior_close:N \g__ztikz_file_ior
}
\cs_generate_variant:Nn \ztikz_file_read_lines_cs:n { e }

% check if code changed (compare hash value of source code)
\seq_new:N \l__ztikz_hash_seq
\cs_new_protected:Npn \ztikz_hash_if_change_cs:n #1 {
    % Param: #1 -> file name of source code
    % read from hash list and remove duplicate item
    \ztikz_file_read_lines_cs:e { ./ztikz_output/ztikz.hash }
    \seq_gremove_duplicates:N \g__ztikz_file_hash_seq
    % check if in
    \file_get_mdfive_hash:nN {#1} \l__ztikz_hash_seq
    \typeout{current~hash~is~\l__ztikz_hash_seq}
    \seq_if_in:NVTF \g__ztikz_file_hash_seq \l__ztikz_hash_seq {
        \bool_gset_false:N \g__hash_change_bool
    }{
        \bool_gset_true:N \g__hash_change_bool
        \typeout{current~hash~is~unique~-->~recorded}
        \sys_shell_now:x { echo~ \l__ztikz_hash_seq >> ./ztikz_output/ztikz.hash}
    }
}
\cs_generate_variant:Nn \ztikz_hash_if_change_cs:n { x }
\end{source}

\subsection{externalize for tikzpicture}
\begin{codeprint}
% set options
\cs_new_protected:Npn \ztikz_define_option:n
    { \keys_define:nn { ztikz / option } }

\ztikz_define_option:n {
    external    .bool_set:N = \l__ztikz_external_bool,
    external    .initial:n  = { true },
}
\ProcessKeysOptions {ztikz / option}

\bool_if:NT \l__ztikz_external_bool {
    \usetikzlibrary{external}
    \tikzexternalize[prefix=ztikz_output/tikz_data/]
}
\end{codeprint}

\subsection{ShowPoint}
\begin{source}
% ==> show point
\keys_define:nn { ztikz / point }{
    type    .str_set:N  = \l__point_type_str,
    type    .initial:n = { * },
    radius  .dim_set:N = \l__point_radius_dim,
    radius  .initial:n = { 1pt },
    color   .tl_set:N  = \l__point_color_tl, 
    color   .initial:n = { black },
    opacity .tl_set:N  = \l__point_opacity_tl, 
    opacity .initial:n = { 1 },
    rotate  .fp_set:N  = \l__point_rotate_angle,
    rotate  .initial:n = { 0 },  
}
\NewDocumentCommand\ShowPoint{ O{}mO{}O{} }{
    \group_begin:
    \keys_set:nn { ztikz / point } { #1 }
    \seq_set_split:Nnn \l__point_list_seq { ; }{#2}
    \seq_set_split:Nnn \l__point_label_seq { ; }{#3}
    \int_step_inline:nnnn {1}{1}{\seq_count:N \l__point_list_seq}{        
        \draw plot [
            only~ marks,
            mark = \str_use:N \l__point_type_str, 
            mark~ size = \dim_use:N \l__point_radius_dim,
            mark~ options = {
                rotate  = \fp_use:N \l__point_rotate_angle, 
                opacity = \tl_use:N \l__point_opacity_tl, 
                color   = \tl_use:N \l__point_color_tl,
                ball~ color = \tl_use:N \l__point_color_tl,
            }
        ] coordinates{\seq_item:Nn \l__point_list_seq{##1}} node[#4]{\seq_item:Nn \l__point_label_seq{##1}};
    }
    \group_end:
}

% ==> show grid
\NewDocumentCommand\ShowGrid{ O{color=gray, very~ thin, step=1}m }{
    \seq_set_split:Nnn \l__grid_param_ii_seq { ; }{#2}
    \draw[#1] \seq_item:Nn \l__grid_param_ii_seq{1}  grid \seq_item:Nn \l__grid_param_ii_seq{2};
}
\end{source}

\subsection{ShowAxis}
\begin{source}
% ==> show axis
\keys_define:nn { ztikz / axis }{
    % basic tick args
    tickStart       .fp_set:N   = \l__start_fp,
    tickStart       .initial:n  = { -5 },
    tickEnd         .fp_set:N   = \l__end_fp,
    tickEnd         .initial:n  = { 5 },
    axisRotate      .fp_set:N   = \l__axis_rotate_angle,
    axisRotate      .initial:n  = { 0 },
    % tick dimension spec
    mainStep        .fp_set:N   = \l__main_step_fp,
    mainStep        .initial:n  = { 1.0 },
    subStep         .fp_set:N   = \l__sub_step_fp,
    subStep         .initial:n  = { 0.1 },
    mainTickLabel   .tl_set:N   = \l__main_tick_label_tl,
    mainTickLabel   .initial:n  = { \fp_use:N {\CurrentFp} },
    tickLabelShift  .dim_set:N  = \l__tick_label_shift_dim,
    tickLabelShift  .initial:n  = { 0pt },
    mainTickLength  .dim_set:N  = \l__main_tick_length_dim,
    mainTickLength  .initial:n  = { 4pt },
    subTickLength   .dim_set:N  = \l__sub_tick_length_dim,
    subTickLength   .initial:n  = { 2pt },
    mainTickLabelPosition .tl_set:N  = \l__main_tick_label_position_tl,
    mainTickLabelPosition .initial:n = { below },
    % color spec
    axisColor       .tl_set:N   = \l__axis_color_tl,
    axisColor       .initial:n  = { black },
    mainTickColor   .tl_set:N   = \l__main_tick_color_tl,
    mainTickColor   .initial:n  = { black },
    subTickColor    .tl_set:N   = \l__sub_tick_color_tl,
    subTickColor    .initial:n  = { black },
    mainTickLabelColor .tl_set:N  = \l__main_tick_label_color_tl,
    mainTickLabelColor .initial:n = { black },
    % tick cross type spec
    tickStyle       .choice:,
    tickStyle/cross .code:n     = \tl_set:Nn \l__tick_spec_tl { cross },
    tickStyle/above .code:n     = \tl_set:Nn \l__tick_spec_tl { above },
    tickStyle/below .code:n     = \tl_set:Nn \l__tick_spec_tl { below },
}
% ticks style
\tl_new:N  \l__tick_type_tl    % `main' or `sub'
\tl_new:N  \l__tick_spec_tl    % `cross', `above' or `below
\tl_new:N  \l__tick_color_tl 
\dim_new:N \l__tick_length_dim
\tl_new:N  \l__node_text_tl
% draw ticks (main or sub)
\cs_new_protected:Npn \ztikz_draw_axis_ticks_cs:n #1 {
    \str_case:NnT \l__tick_type_tl {
        {main}{
            \dim_set_eq:NN \l__tick_length_dim\l__main_tick_length_dim
            \tl_set:NV \l__tick_color_tl\l__main_tick_color_tl
            \tl_set:Nn \l__node_text_tl {\tl_use:N \l__main_tick_label_tl}
        }
        {sub}{
            \dim_set_eq:NN \l__tick_length_dim\l__sub_tick_length_dim
            \tl_set:NV \l__tick_color_tl \l__sub_tick_color_tl
            \tl_set:Nn \l__node_text_tl {}
        }
    }{}
    \str_case:VnT \l__tick_spec_tl {
        {cross}{
            \draw[\tl_use:N \l__tick_color_tl] 
                (#1, 0)++(0, \dim_eval:n {\l__tick_length_dim/2}) -- ++(0, \dim_eval:n {-\l__tick_length_dim}) 
                node[\tl_use:N \l__main_tick_label_position_tl] 
                {\textcolor{\tl_use:N \l__main_tick_label_color_tl}{\tl_use:N \l__node_text_tl}};
        }
        {above}{
            \draw[\tl_use:N \l__tick_color_tl] (#1, 0) -- ++(0, \dim_eval:n {\l__tick_length_dim/2}) 
                node[\tl_use:N \l__main_tick_label_position_tl] 
                {\textcolor{\tl_use:N \l__main_tick_label_color_tl}{\tl_use:N \l__node_text_tl}};
        }
        {below}{
            \draw[\tl_use:N \l__tick_color_tl] (#1, 0) -- ++(0, \dim_eval:n {-\l__tick_length_dim/2}) 
                node[\tl_use:N \l__main_tick_label_position_tl=\dim_use:N \l__tick_label_shift_dim] 
                {\textcolor{\tl_use:N \l__main_tick_label_color_tl}{\tl_use:N \l__node_text_tl}};
        }
    }{}
} 
% draw axis
\int_new:N \l__substep_num_int
\fp_new:N \CurrentFp
\NewDocumentCommand\ShowAxis{O{}m}{
        \group_begin:
        \keys_set:nn { ztikz / axis } { #1 }
        \seq_set_split:Nnn \l__points_seq { ; }{#2}
        \begin{scope}[rotate=\fp_use:N \l__axis_rotate_angle]
        \draw[->, \tl_use:N \l__axis_color_tl] \seq_item:Nn \l__points_seq{1} -- \seq_item:Nn \l__points_seq{2};
        % draw ticks
        \fp_step_inline:nnnn {\fp_eval:n {\l__start_fp}}{\fp_use:N \l__main_step_fp}{\fp_use:N \l__end_fp}{
            % main ticks
            \tl_set:Nn \l__tick_type_tl {main}
            \fp_gset:Nn \CurrentFp {##1}
            \ztikz_draw_axis_ticks_cs:n {##1}
            % sub ticks
            \tl_set:Nn \l__tick_type_tl {sub}
            \int_set:Nn \l__substep_num_int {\fp_eval:n {floor(\l__main_step_fp/\l__sub_step_fp)}}
            \fp_compare:nNnTF {##1}<{\fp_eval:n {floor(\l__end_fp)}}{
                \fp_step_function:nnnN                 
                    {\fp_eval:n {##1+\l__sub_step_fp}}
                    {\fp_use:N \l__sub_step_fp}
                    {\fp_eval:n {##1+\l__substep_num_int*\l__sub_step_fp}}
                    \ztikz_draw_axis_ticks_cs:n
            }{}
        }
        \end{scope}
        \group_end:
}
\NewDocumentCommand{\xAxis}{O{-2}O{8}}{
    \ShowAxis[
        tickStart=\fp_eval:n {#1+1}, tickEnd=\fp_eval:n {#2-0.75}, 
        mainStep=1, subStep=.25, 
        axisRotate=0, axisColor=black,
        mainTickColor=black, subTickColor=black,
        mainTickLenght=10pt, subTickLenght=5pt,
        tickLabelShift=0pt, tickStyle=below, 
        mainTickLabelPosition=below
    ]{(#1, 0); (#2, 0)}
}
\NewDocumentCommand{\yAxis}{O{-2}O{8}}{
    \ShowAxis[
        tickStart=\fp_eval:n {#1+1}, tickEnd=\fp_eval:n {#2-0.75}, 
        mainStep=1, subStep=.25, 
        axisRotate=90, axisColor=black,
        mainTickColor=black, subTickColor=black,
        mainTickLenght=10pt, subTickLenght=5pt,
        tickLabelShift=0pt, tickStyle=above, 
        mainTickLabelPosition=left
    ]{(#1, 0); (#2, 0)}
}

% ==> show intersection
\NewDocumentCommand\ShowIntersection{ omm }{
    \seq_set_split:Nnn \l__intersection_num_seq { ; }{#2}
    % get all intersections
    \path[name~ intersections={of=\seq_item:Nn \l__intersection_num_seq{1}~ and~ \seq_item:Nn \l__intersection_num_seq{2}}]; 
    % show all intersections by \ShowPoint
    \int_step_inline:nnnn {1}{1}{#3}{
        \ShowPoint[#1]{(intersection-##1)}
    }
}
\end{source}

\subsection{Polygon}
\begin{source}
% ==> polygon
\keys_define:nn { ztikz / polygon }{
    radius       .fp_set:N  = \l__polygon_radius_fp,
    radius       .initial:n = { 1 },
    edgeColor    .tl_set:N  = \l__polygon_edge_color_tl,
    edgeColor    .initial:n = { black },
    fillColor    .tl_set:N  = \l__polygon_fill_color_tl,
    fillColor    .initial:n = { white },
    fillOpacity  .fp_set:N  = \l__polygon_fill_opacity_fp,
    fillOpacity  .initial:n = { 0 },
    rotate       .fp_set:N  = \l__polygon_rotate_angle,
    rotate       .initial:n = { 0 },
    shift        .tl_set:N  = \l__polygon_shift_tl,
    shift        .initial:n = { (0,0) },
    marker       .tl_set:N  = \l__polygon_marker_option_tl,
    marker       .initial:n = { },
}
\tl_new:N \l__poly_path_tl
\NewDocumentCommand\Polygon{ O{}m }{
    \group_begin:
    \keys_set:nn { ztikz / polygon } { #1 }
    % strip '(' and ')'
    \tl_replace_once:Nnn \l__polygon_shift_tl{(}{}
    \tl_replace_once:Nnn \l__polygon_shift_tl{)}{}
    \coordinate (mv) at (\tl_use:N \l__polygon_shift_tl);
    % create polygon
    \begin{scope}[shift=(mv), rotate=\fp_use:N \l__polygon_rotate_angle]
    \int_step_inline:nnn {1}{#2}{
        % draw edges
        \fp_set:Nn \l_angle_fp {360/#2*##1*\c_one_degree_fp}
        \fp_set:Nn \l_angle_next_fp {360/#2*(##1+1)*\c_one_degree_fp}
        \draw[\tl_use:N \l__polygon_edge_color_tl] (\fp_eval:n {\l__polygon_radius_fp*cos(\l_angle_fp)},      \fp_eval:n {\l__polygon_radius_fp*sin(\l_angle_fp)})
            --(\fp_eval:n {\l__polygon_radius_fp*cos(\l_angle_next_fp)}, \fp_eval:n {\l__polygon_radius_fp*sin(\l_angle_next_fp)});
        % fill polygon path 
        \int_compare:nNnTF {##1}<{#2}{
            \tl_put_right:Nn \l__poly_path_tl {(p##1)--}
        }{
            \tl_put_right:Nn \l__poly_path_tl {(p##1)--cycle}
        }
        % mark coordinates
        \coordinate (p##1) at (\fp_eval:n {\l__polygon_radius_fp*cos(\l_angle_fp)}, \fp_eval:n {\l__polygon_radius_fp*sin(\l_angle_fp)});
    }
    % fill polygon
    \fill [\tl_use:N \l__polygon_fill_color_tl, fill~opacity=\fp_use:N \l__polygon_fill_opacity_fp] \l__poly_path_tl;
    % show markers
    \int_step_inline:nnn {1}{#2}{
        \ShowPoint[\l__polygon_marker_option_tl]{(p##1)}
    }
    \end{scope} 
    \group_end:
}
\end{source}


\subsection{Plot precise}
\begin{source}
% change precise of each plot type
\NewDocumentCommand\PlotPrecise{ mO{once}m }{
    \str_if_eq:nnTF {#2}{once}{
        \bool_gset_true:c {g__#1_precise_bool}
    }{\relax}
    % check plot type
    \str_case:nnF {#1}{
        {plot}{
            \sys_shell_now:x {sed~ -i~ "3s|set~ samples~ .*|set~ samples~ #3|"~             \g__ztikz_scripts_path_tl/plot.gp       }
        }
        {param}{
            \sys_shell_now:x {sed~ -i~ "3s|set~ samples~ .*|set~ samples~ #3|"~             \g__ztikz_scripts_path_tl/param_plot.gp }
        }
        {polar}{
            \sys_shell_now:x {sed~ -i~ "3s|set~ samples~ .*|set~ samples~ #3|"~             \g__ztikz_scripts_path_tl/param_plot.gp }
        }
        {contour}{
            \sys_shell_now:x {sed~ -i~ "2s|set~ samples~ .*|set~ samples~ #3,#3|"~          \g__ztikz_scripts_path_tl/contour_plot.gp}
            \sys_shell_now:x {sed~ -i~ "3s|set~ isosamples~ .*|set~ isosamples~ #3,#3|"~    \g__ztikz_scripts_path_tl/contour_plot.gp}
        }
    }{
        \msg_new:nnn {ztikz}{ztikz-plot-type}{plot~type~support~is:~'plot',~'param',~'polar',~'contour'}
        \msg_error:nn {ztikz}{ztikz-plot-type}
    }
}
\end{source}

\subsection{curve plot}
\begin{source}
% escape shell commands 
\cs_generate_variant:Nn \sys_shell_mv:nn {xx}
\cs_generate_variant:Nn \sys_shell_now:n {x}

% gnu data reference
\NewDocumentCommand\gnudata{m}{
    \tl_use:N \g__ztikz_gnu_path_tl/gnu_data_ \int_use:N \g__tikz_env_index_int _#1.table
}

% gnuplot data rename, plot and precise reset
\cs_new_protected:Npn \ztikz_gnu_data_plot_cs:nnn #1#2#3 {
    % rename data file
    \int_gadd:Nn \g__gnu_data_index_int {1}
    \tl_set:Nx \l__gnu_data_new_name_tl {
        gnu_data_\int_use:N \g__tikz_env_index_int _
        \int_use:N \g__gnu_data_index_int.table
    }
    \tl_set:Nx \l__gnu_data_full_path_tl {\g__ztikz_gnu_path_tl/\l__gnu_data_new_name_tl}
    \sys_shell_mv:xx {\g__ztikz_gnu_path_tl/gnu_data.table}
                    {\l__gnu_data_full_path_tl}
    % plot data file
    \tl_if_empty:nTF {#3}{
        \draw[#2] plot[smooth] file {\l__gnu_data_full_path_tl};
    }{
        \group_begin:
        \keys_set:nn { ztikz / point } { #3 }
        \draw plot [
            mark = \str_use:N \l__point_type_str, 
            mark~ size = \dim_use:N \l__point_radius_dim,
            mark~ options = {
                rotate  = \fp_use:N \l__point_rotate_angle, 
                opacity = \tl_use:N \l__point_opacity_tl, 
                color   = \tl_use:N \l__point_color_tl,
                ball~ color = \tl_use:N \l__point_color_tl,
            }
        ] file {\l__gnu_data_full_path_tl};
        \group_end:
    }
    % reset precise (default 300 for plot precise)
    \bool_if:cTF {g__#1_precise_bool}{
        \PlotPrecise{#1}{300}
    }{\relax}
}

% ==> simple 2d function
\NewDocumentCommand\Plot{ O{-5:5}O{color=black}O{}m }{
    % sed gnuplot scripts
    \sys_shell_now:x {sed~ -i~ "8s|set~ xr~ .*|set~ xr~ [#1]|"~ \g__ztikz_scripts_path_tl/plot.gp}
    \sys_shell_now:x {sed~ -i~ "7s|f(x)~ =~ .*|f(x)~ =~ #4|"  ~ \g__ztikz_scripts_path_tl/plot.gp}
    \sys_shell_now:x {gnuplot~                                  \g__ztikz_scripts_path_tl/plot.gp}
    % plot data, reset precise
    \ztikz_gnu_data_plot_cs:nnn {plot}{#2}{#3}
}

% ==> implicit 2d function
\NewDocumentCommand\ContourPlot{ O{-5:5; -5:5}O{color=black}O{}m }{
	% split the first param
    \seq_set_split:Nnn \l__plot_domain_seq { ; }{#1}
    \tl_set:Nn \l__y_domain_tl {\seq_item:Nn \l__plot_domain_seq{2}}
    \exp_args:Nx \tl_if_blank:nTF {\seq_item:Nn \l__plot_domain_seq{2}}{
        \tl_set:Nn \l__y_domain_tl {*:*}
    }{
        \tl_set:Nn \l__y_domain_tl {\seq_item:Nn \l__plot_domain_seq{2}}
    }
    % sed gnuplot scripts
    \sys_shell_now:x {sed~ -i~ "11s|set~ xr~ .*|set~ xr~ [\seq_item:Nn \l__plot_domain_seq{1}]|"~ \g__ztikz_scripts_path_tl/contour_plot.gp}
    \sys_shell_now:x {sed~ -i~ "12s|set~ yr~ .*|set~ yr~ [\tl_use:N \l__y_domain_tl]|"~           \g__ztikz_scripts_path_tl/contour_plot.gp}
    \sys_shell_now:x {sed~ -i~ "14s|f(x,~ y)~ =~ .*|f(x,~ y)~ =~ #4|"~                            \g__ztikz_scripts_path_tl/contour_plot.gp}
    \sys_shell_now:x {gnuplot~                                                                    \g__ztikz_scripts_path_tl/contour_plot.gp}
    % plot data, reset precise
    \ztikz_gnu_data_plot_cs:nnn {contour}{#2}{#3}
}

% ==> parametric 2d function
\NewDocumentCommand\ParamPlot{ O{0:2*pi}O{color=black}O{}m }{
    % sed gnuplot scripts
    \sys_shell_now:x {sed~ -i~ "8s|set~ trange~ .*|set~ trange~ [#1]|"~     \g__ztikz_scripts_path_tl/param_plot.gp}
    \sys_shell_now:x {sed~ -i~ "9s|plot~ .*|plot~ #4|"~                     \g__ztikz_scripts_path_tl/param_plot.gp}
    \sys_shell_now:x {gnuplot~                                              \g__ztikz_scripts_path_tl/param_plot.gp}
    % plot data, reset precise
    \ztikz_gnu_data_plot_cs:nnn {param}{#2}{#3}
}

% ==> polar 2d function
\NewDocumentCommand\PolarPlot{ O{0:2*pi}O{color=black}O{}m }{
    % sed gnuplot scripts
    \sys_shell_now:x {sed~ -i~ "8s|set~ trange~ .*|set~ trange~ [#1]|"~     \g__ztikz_scripts_path_tl/polar_plot.gp}
    \sys_shell_now:x {sed~ -i~ "9s|plot~ .*|plot~ #4|"~                     \g__ztikz_scripts_path_tl/polar_plot.gp}
    \sys_shell_now:x {gnuplot~                                              \g__ztikz_scripts_path_tl/polar_plot.gp}
    % plot data, reset precise
    \ztikz_gnu_data_plot_cs:nnn {polar}{#2}{#3}
}
\end{source}

\subsection{Statistic Plot}
\begin{source}
% ==> statistic plot function
\cs_new_protected:Npn \ztikz_statistic_plot_cs:nnnn #1#2#3#4 {
    \tl_if_empty:nTF {#3}{
        \draw[#2] plot[#1] file {#4};
    }{
        \group_begin:
        \keys_set:nn { ztikz / point } { #3 }
        \draw[#2] plot [
            % stairs options
            #1,
            % marker options
            mark = \str_use:N \l__point_type_str, 
            mark~ size = \dim_use:N \l__point_radius_dim,
            mark~ options = {
                rotate  = \fp_use:N \l__point_rotate_angle, 
                opacity = \tl_use:N \l__point_opacity_tl, 
                color   = \tl_use:N \l__point_color_tl,
                ball~ color = \tl_use:N \l__point_color_tl,
            }
        ] file {#4};
        \group_end:
    }
}
\cs_generate_variant:Nn \ztikz_statistic_plot_cs:nnnn {xnnn}

\tl_new:N \l__tempa_tl
\tl_new:N \l__tempb_tl
\seq_new:N \l__statistic_option_tl
% 1. \ListPlot --> set opcity=0 in the above Plot commands
% 2. stairs plot
\NewDocumentCommand\StairsPlot{ O{plot-left;jump-left}O{color=black}O{}m }{
    \seq_set_split:Nnn \l__statistic_option_tl { ; }{#1}
    \str_case:enF {\seq_item:Nn \l__statistic_option_tl{1}}{
        {plot-left}{\tl_set:Nn \l__tempa_tl {const~plot~mark~left}}
        {plot-right}{\tl_set:Nn \l__tempa_tl {const~plot~mark~right}}
        {plot-mid}{\tl_set:Nn \l__tempa_tl {const~plot~mark~mid}}
        {}{\tl_set:Nn \l__tempa_tl {}}
    }{ 
        \msg_new:nnn {ztikz}{ztikz-stairs-plot}{current~stairs~plot~type~is:~'#1'~,~ invalide}
        \msg_error:nn {ztikz}{ztikz-stairs-plot}
    }
    \str_case:enF {\seq_item:Nn \l__statistic_option_tl{2}}{
        {jump-left}{\tl_set:Nn \l__tempb_tl {jump~mark~left}}
        {jump-right}{\tl_set:Nn \l__tempb_tl {jump~mark~right}}
        {jump-mid}{\tl_set:Nn \l__tempb_tl {jump~mark~mid}}
        {}{\tl_set:Nn \l__tempb_tl {}}
    }{ 
        \msg_new:nnn {ztikz}{ztikz-stairs-plot}{current~stairs~jump~type~is:~'#1'~,~ invalide}
        \msg_error:nn {ztikz}{ztikz-stairs-plot}
    }
    \ztikz_statistic_plot_cs:xnnn {\l__tempa_tl,\l__tempb_tl}{#2}{#3}{#4}
}
% 3. stem plot
\NewDocumentCommand\StemPlot{ O{x}O{color=black}O{}m }{
    \str_case:enF {#1}{
        {x}{\tl_set:Nn \l__tempa_tl {ycomb}}
        {y}{\tl_set:Nn \l__tempa_tl {xcomb}}
        {o}{\tl_set:Nn \l__tempa_tl {polar~ comb}}
        {}{\tl_set:Nn \l__tempa_tl  {ycomb}}
    }{ 
        \msg_new:nnn {ztikz}{ztikz-stem-plot}{current~stem~plot~type~is:~'#1'~,~ invalide}
        \msg_error:nn {ztikz}{ztikz-stem-plot}
    }
    \ztikz_statistic_plot_cs:xnnn {\l__tempa_tl}{#2}{#3}{#4}
}
% 4. bar plot
\NewDocumentCommand\BarPlot{ O{ybar}O{color=black}O{}m }{
    \str_case:enF {#1}{
        {x}{\tl_set:Nn \l__tempa_tl {ybar}}
        {y}{\tl_set:Nn \l__tempa_tl {xbar}}
        {xc}{\tl_set:Nn \l__tempa_tl {ybar~ interval}}
        {yc}{\tl_set:Nn \l__tempa_tl {xbar~ interval}}
        {}{\tl_set:Nn \l__tempa_tl  {ybar}}
    }{ 
        \msg_new:nnn {ztikz}{ztikz-bar-plot}{current~bar~plot~type~is:~'#1'~,~ invalide}
        \msg_error:nn {ztikz}{ztikz-bar-plot}
    }
    \ztikz_statistic_plot_cs:xnnn {\l__tempa_tl}{#2}{#3}{#4}
}
\end{source}

\subsection{ShadePlot}
\begin{source}
% 5. shade plot
\NewDocumentCommand\ShadePlot{ O{defaultMode}O{2pt}m }{
    \path [shading=#1, shading~ path={draw=transparent!0, line~ width=#2}] plot file {#3};
}

% ==> shade curve draw
\ExplSyntaxOff
% -----------------------------------------------------------------------------------------------
% https://tex.stackexchange.com/questions/497977/tikz-draw-multicolor-curve-with-smooth-gradient
% -----------------------------------------------------------------------------------------------
\newif\iftikz@shading@path
\tikzset {
    fading xsep/.store in=\pgfpathfadingxsep,
    fading ysep/.store in=\pgfpathfadingysep,
    fading sep/.style={fading xsep=#1, fading ysep=#1},
    fading sep=0.0cm,
    shading path/.code= {
        % Prevent this stuff happning recursively.
        \iftikz@shading@path
        \else
            \tikz@shading@pathtrue
            % \tikz@addmode installs the `modes' (e.g., fill, draw, shade) 
            % to be applied to the path. It isn't usualy for doing more
            % changes to the path's construction.
            \tikz@addmode{
                \pgfgetpath\pgf@currentfadingpath%
                % Get the boudning box of the current path size including the fading sep
                \pgfextract@process\pgf@fadingpath@southwest{\pgfpointadd{\pgfqpoint{\pgf@pathminx}{\pgf@pathminy}}
                    {\pgfpoint{-\pgfpathfadingxsep}{-\pgfpathfadingysep}}}%
                \pgfextract@process\pgf@fadingpath@northeast{\pgfpointadd{\pgfqpoint{\pgf@pathmaxx}{\pgf@pathmaxy}}
                    {\pgfpoint{\pgfpathfadingxsep}{\pgfpathfadingysep}}}
                % Clear the path
                \pgfsetpath\pgfutil@empty%                         
                % Interrupt the path and picture to create a fading.
                \pgfinterruptpath
                \pgfinterruptpicture
                \begin{tikzfadingfrompicture}[name=.]
                    \path [shade=none, fill=none, #1] \pgfextra {
                        % Set the softpath. Any transformations in #1 will have no effect.
                        % This will *not* update the bounding box...
                        \pgfsetpath\pgf@currentfadingpath
                        % ...so it is done manually.
                        \pgf@fadingpath@southwest
                        \expandafter\pgf@protocolsizes{\the\pgf@x}{\the\pgf@y}%
                        \pgf@fadingpath@northeast%
                        \expandafter\pgf@protocolsizes{\the\pgf@x}{\the\pgf@y}%
                    };
                    % Now get the bounding of the picture.
                    \xdef\pgf@fadingboundingbox@southwest{\noexpand\pgfqpoint{\the\pgf@picminx}{\the\pgf@picminy}}%
                    \xdef\pgf@fadingboundingbox@northeast{\noexpand\pgfqpoint{\the\pgf@picmaxx}{\the\pgf@picmaxy}}%
                \end{tikzfadingfrompicture}
                \endpgfinterruptpicture
                \endpgfinterruptpath
                % Install a rectangle that covers the shaded/faded path picture.                                
                \pgfpathrectanglecorners{\pgf@fadingboundingbox@southwest}{\pgf@fadingboundingbox@northeast}
                % Make the fading happen.
                \def\tikz@path@fading{.}
                \tikz@mode@fade@pathtrue
                \tikz@fade@adjustfalse%10p
                % Shift the fading to the mid point of the rectangle
                \pgfpointscale{0.5}{\pgfpointadd{\pgf@fadingboundingbox@southwest}{\pgf@fadingboundingbox@northeast}}
                \edef\tikz@fade@transform{shift={(\the\pgf@x,\the\pgf@y)}}
            }
        \fi
    }
}
\def\ztikzShadeMode#1#2#3{
    \pgfutil@tempcnta=0\relax
    \pgfutil@for\pgf@tmp:={#3}\do{\advance\pgfutil@tempcnta by1}
    \ifnum\pgfutil@tempcnta=1\relax
        \edef\pgf@spec{color(0)=(#3);color(100)=(#3)}
    \else
        \pgfmathparse{50/(\pgfutil@tempcnta-1)}\let\pgf@step=\pgfmathresult
        \pgfutil@tempcntb=1\relax
        \pgfutil@for\pgf@tmp:={#3}\do{
            \ifnum\pgfutil@tempcntb=1\relax
            \edef\pgf@spec{color(0)=(\pgf@tmp);color(25)=(\pgf@tmp)}
            \else
            \ifnum\pgfutil@tempcntb<\pgfutil@tempcnta\relax
            \pgfmathparse{25+\pgf@step/4+(\pgfutil@tempcntb-1)*\pgf@step}
            \edef\pgf@spec{\pgf@spec;color(\pgfmathresult)=(\pgf@tmp)}
            \else
            \edef\pgf@spec{\pgf@spec;color(75)=(\pgf@tmp);color(100)=(\pgf@tmp)}
            \fi
            \fi
            \advance\pgfutil@tempcntb by1\relax
        }
    \fi
    \csname pgfdeclare#2shading\endcsname{#1}{100}\pgf@spec
}
\ztikzShadeMode{defaultMode}{horizontal}{white,black}
\ExplSyntaxOn
\end{source}


\section{Python}
\subsection{matplotlib}
\begin{source}
\cs_generate_variant:Nn \xsim_file_write_start:nn {nx}
\cs_generate_variant:Nn \sys_shell_mv:nn {xx}
% ==> python-matplotlib
\NewDocumentEnvironment{pyfig}{ O{width=.75\linewidth}m }{
    \newcommand{\pyfile}{#2}
    \xsim_file_write_start:nx {\c_true_bool}{\g__ztikz_python_path_tl/#2}
    }{ 
    \xsim_file_write_stop:
    % step picture index
    \int_gadd:Nn \g__picture_index_int {1}   
    % check if hash changed
    \ztikz_hash_if_change_cs:x {\g__ztikz_python_path_tl/\pyfile}   
    \bool_if:NTF \g__hash_change_bool {
        % add save figure to source
        \sys_if_platform_windows:TF {
            \exp_args:Nx \sys_shell_now:n {
                echo~ plt.savefig('\g__ztikz_python_path_tl/\pyfile.pdf')~ >>~ \g__ztikz_python_path_tl/\pyfile
            }
        }{
            \exp_args:Nx \sys_shell_now:n {
                echo~ "plt.savefig('\g__ztikz_python_path_tl/\pyfile.pdf')"~ >>~ \g__ztikz_python_path_tl/\pyfile
            }
        }
        % excute python source
        \exp_args:Nx \sys_shell_now:n {python~ \g__ztikz_python_path_tl/\pyfile} 
        \includegraphics[#1]{\g__ztikz_python_path_tl/\pyfile.pdf}
        \typeout{Writing~ 'pyfig'~environment~source~to~\tl_use:N \g__ztikz_python_path_tl/\pyfile}
    }{
        \includegraphics[#1]{\g__ztikz_python_path_tl/\pyfile.pdf}
        \typeout{skip~recompile~by~python,~using~the~cache~picture~\int_use:N \g__picture_index_int}
    }
}    
\end{source}

\subsection{pycode}
\begin{source}
% ==> python-code-env
\NewDocumentEnvironment{pycode}{ m }{
    \newcommand{\pyfile}{#1}
    \xsim_file_write_start:nx {\c_true_bool}{\g__ztikz_python_path_tl/#1}
    }{ 
    \xsim_file_write_stop:
    % step picture index
    \int_gadd:Nn \g__picture_index_int {1}   
    % check if hash changed
    \ztikz_hash_if_change_cs:x {\g__ztikz_python_path_tl/\pyfile}   
    \bool_if:NTF \g__hash_change_bool {
        % excute python source
        \exp_args:Nx \sys_shell_now:n {python~ \g__ztikz_python_path_tl/\pyfile} 
        \input{\g__ztikz_python_path_tl/\pyfile.out}
        \typeout{Writing~ 'pycode'~environment~source~to~\tl_use:N \g__ztikz_python_path_tl/\pyfile}
    }{
        \input{\g__ztikz_python_path_tl/\pyfile.out}
        \typeout{skip~recompile~by~python,~using~the~cache~pycode~result~\int_use:N \g__picture_index_int}
    }
}
\end{source}

\subsection{sympy}
\begin{source}
% ==> python-sympy
\NewDocumentCommand\sympy{m}{
    % step sympy result index
    \int_gadd:Nn \g__sympy_index_int {1}  
    \tl_set:Nx \l__current_sympy_index_tl {\int_use:N \g__sympy_index_int}
    % sympy source write
    \sys_shell_now:x {sed~ -i~ "8s|F_res~ =~ .*|F_res~ =~  #1|"~ \g__ztikz_scripts_path_tl/sympy_script.py}
    % check hash
    \ztikz_hash_if_change_cs:x {\g__ztikz_scripts_path_tl/sympy_script.py}
    \bool_if:NTF \g__hash_change_bool {
        \sys_shell_now:x {python~ \g__ztikz_scripts_path_tl/sympy_script.py}
        \sys_shell_mv:xx
            {\g__ztikz_python_path_tl/sympy.out}
            {\g__ztikz_python_path_tl/sympy_\int_use:N \g__sympy_index_int .out}
        \typeout{using~python~sympy~calculating~question~\l__current_sympy_index_tl ...}
        \exp_args:Nx \input{\g__ztikz_python_path_tl/sympy_\l__current_sympy_index_tl.out}
    }{
        \exp_args:Nx \input{\g__ztikz_python_path_tl/sympy_\l__current_sympy_index_tl.out}
        \typeout{skip~recompile,~using~the~cache~sympy~result~\l__current_sympy_index_tl}
    }
}
\end{source}

\subsection{python eval}
\begin{source}
% read from external file 
\cs_new_protected:Npn \zlatex_Readlines_cs:nn #1#2 {
    \ior_open:Nn \g__file_read_ior {#2}
    \str_case:nnF {#1}{
        {hold}{
            \ior_get:NN \g__file_read_ior \g__file_content_tl
        }
        {str}{
            \ior_str_get:NN \g__file_read_ior \g__file_content_tl
        }
    }{}
    \tl_use:N \g__file_content_tl
}
\cs_generate_variant:Nn \zlatex_Readlines_cs:nn {xx}

% ==> 1-line python command
\NewDocumentCommand\py{O{hold}m}{
    % sympy source write
    \sys_shell_now:x {sed~ -i~ "6s|Float_res~ =~ .*|Float_res~ =~ #2|"~ \g__ztikz_scripts_path_tl/python_script.py}
    % calculation
    \typeout{using~python~float~module~calculating...}
    \sys_shell_now:x {python~ \g__ztikz_scripts_path_tl/python_script.py}
    % using \ior_get: 
    \zlatex_Readlines_cs:xx {#1}{\g__ztikz_python_path_tl/PyFloat.out}
    % ---> cause bug that can't write ToC to file
    % \iow_close:N \g__file_read_ior
}
\end{source}





\section{Mathematica}
\subsection{picture}
\begin{source}
% wolfram graphicx
\NewDocumentEnvironment{mmafig}{ O{width=.75\linewidth}m }{
    \newcommand{\mmafile}{#2}
    \xsim_file_write_start:nx {\c_true_bool}{\g__ztikz_mma_path_tl/#2}
    }{ 
    \xsim_file_write_stop:
    % step picture index
    \int_gadd:Nn \g__picture_index_int {1}   
    % check if hash changed
    \ztikz_hash_if_change_cs:x {\g__ztikz_mma_path_tl/\mmafile}   
    \bool_if:NTF \g__hash_change_bool {
        % excute mathematica script
        \exp_args:Nx \sys_shell_now:n {wolframscript~ -script~ \g__ztikz_mma_path_tl/\mmafile} 
        \includegraphics[#1]{\g__ztikz_mma_path_tl/\mmafile.pdf}
        \typeout{Writing~ 'mmafig'~environment~source~to~\tl_use:N \g__ztikz_mma_path_tl/\mmafile}
    }{
        \includegraphics[#1]{\g__ztikz_mma_path_tl/\mmafile.pdf}
        \typeout{skip~recompile~by~wolframscript,~using~the~cache~picture~\int_use:N \g__picture_index_int}
    }
}
\end{source}


\subsection{Symbols Calculation}
\begin{source}
% input result of wolfram
\cs_new_protected:Nn \ztikz_wolfram_input_result_cs: {
    % write export command
    \iow_now:Nx \g_tmpa_iow {Export["\g__ztikz_mma_path_tl/mma_res_\int_use:N \g__mma_index_int.txt", TeXResult]}
    \iow_close:N \g_tmpa_iow
    % calculate and input
    \ztikz_hash_if_change_cs:x {\g__ztikz_mma_path_tl/mma_calc_\int_use:N \g__mma_index_int.wls}   
    \bool_if:NTF \g__hash_change_bool {
        \sys_shell_now:x {wolframscript~ -script~ \g__ztikz_mma_path_tl/mma_calc_\int_use:N \g__mma_index_int.wls}
        \exp_args:Nx \input{\g__ztikz_mma_path_tl/mma_res_\int_use:N \g__mma_index_int.txt}
        \typeout{using~wolframscript~calculating~question~\int_use:N \g__mma_index_int ...}
    }{
        \exp_args:Nx \input{\g__ztikz_mma_path_tl/mma_res_\int_use:N \g__mma_index_int.txt}
        \typeout{skip~recompile,~using~the~cache~wolframscript~result~\int_use:N \g__mma_index_int}
    }
    % step counter, add export command
    \int_gadd:Nn \g__mma_index_int {1}  
}

% wolfram code
\cs_generate_variant:Nn \iow_open:Nn {Nx}
\NewDocumentCommand\wolfram{O{tex}m}{
    % write mathamatica code
    \iow_open:Nx \g_tmpa_iow {\g__ztikz_mma_path_tl/mma_calc_\int_use:N \g__mma_index_int.wls}
    \str_case:nnF {#1}{
        {tex} {
            \iow_now:Nx \g_tmpa_iow { TeXResult = ToString[TeXForm[#2]]; }
        }
        {text} {
            \iow_now:Nx \g_tmpa_iow { TeXResult = ToString[#2]; }
        }
    }{\relax}
    % cache and input result
    \ztikz_wolfram_input_result_cs:
}
\end{source}

\subsection{equation Solve}
\begin{source}
% equation solve
\NewDocumentCommand\wolframsolve{O{part}mO{}O{}}{ 
    % write mathamatica code
    \iow_open:Nx \g_tmpa_iow {\g__ztikz_mma_path_tl/mma_calc_\int_use:N \g__mma_index_int.wls}
    \str_case:nnF {#1}{
        {part} {
            \iow_now:Nx \g_tmpa_iow {
                TeXResult = StringReplace[Row[Solve[#2, {#3}, #4]//Flatten, ","]//TeXForm//ToString, "\\to"->"="];
            }
        }
        {full} {
            \iow_now:Nx \g_tmpa_iow {
                TeXResult = StringReplace[Row[Solve[#2]//Flatten, ","]//TeXForm//ToString, "\\to"->"="];
            }
        }
    }{\relax}
    % cache and input result
    \ztikz_wolfram_input_result_cs:
}

% differential equation solve
\NewDocumentCommand\wolframdsolve{O{part}mO{}O{}}{
    % write mathamatica code
    \iow_open:Nx \g_tmpa_iow {\g__ztikz_mma_path_tl/mma_calc_\int_use:N \g__mma_index_int.wls}
    \str_case:nnF {#1}{
        {part} {
            \iow_now:Nx \g_tmpa_iow {
                TeXResult = StringReplace[Row[DSolve[#2, #3, #4]//Flatten, ","]//TeXForm//ToString, "\\to"->"="];
            }
        }
        {full} {
            \iow_now:Nx \g_tmpa_iow {
                TeXResult = StringReplace[Row[DSolve[#2]//Flatten, ","]//TeXForm//ToString, "\\to"->"="];
            }
        }
    }{\relax}
    % cache and input result
    \ztikz_wolfram_input_result_cs:
}
\end{source}